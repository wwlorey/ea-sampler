\documentclass[11pt]{article}
\usepackage[left=3cm, right=3cm, top=2cm]{geometry}
\usepackage{graphicx}
\usepackage{float}

% Prevent heading numbers
\setcounter{secnumdepth}{0}

% Double spaced
\linespread{1.5}

% Command definitions 
\newcommand{\fitnessplotcaption}[1]{\caption{Evaluations versus Average Local Fitness and Evaluations versus 
    Local Best Fitness for the \textbf{{#1}}, Averaged Over All Runs}}

\newcommand{\addgraphic}[1]{\centerline{\includegraphics[scale=0.6]{report/figures/{#1}.png}}}

\newcommand{\tablecaption}[1]{\caption{Statistical Analysis performed on the {#1}, EA configurations}}

\begin{document}

% Header Page
\thispagestyle{empty}
\begin{center}
\begin{minipage}{0.75\linewidth}
\centering

\vspace{6.2cm}

{\Large CpE5310 Computational Intelligence\par}
\vspace{0.2cm}

{\large Evolutionary Algorithm Sampling\par}
\vspace{3cm}

{\large Prepared by: Nicholas Covert, William Lorey, and Ruicheng Xu\par}
\vspace{0.2cm}

{\large Prepared for: Dr. Steven Corns}
\vspace{3cm}

{\large \today}

\end{minipage}
\end{center}

\clearpage

\tableofcontents

\clearpage


\section{Abstract}

TODO


\section{Random Search}

\subsection{Introduction}

TODO


\subsection{Discussion}

TODO


\begin{figure}[H]
    \addgraphic{random_gen_soln_graph}
    \caption{Evaluations vs. Fitness for Randomly Generated Puzzles (without enforcing black cell number constraint)}
    \label{fig:rand_search_rand}
\end{figure}

\begin{figure}[H]
    \addgraphic{website_puzzle_soln_graph}
    \caption{Evaluations vs. Fitness for Provided Puzzle (without enforcing black cell number constraint)}
    \label{fig:rand_search_web}
\end{figure}


\section{Standard EA}

\subsection{Introduction}

TODO


\subsection{Statistical Analysis Overview}

For both puzzle varieties (the provided puzzle and the generated puzzle) the 
evolutionary algorithm was compared to the random search algorithm to determine 
statistical superiority between the methods, in the context of this problem instance. 
The numerical analysis for each of these experiment classes can be found in the 
Statistical Analysis for Provided Puzzle and Statistical Analysis for Randomly 
Generated Puzzle sections.

In each case, since the distribution was not known to be normal and since the 
sample size was greater than 29, the variances were compared using the f-test. 
The f-test for both experiment comparisons yielded the conclusion that variances 
could be assumed equal. This was because the conditions 

\begin{center}
\texttt{|mean(Random Search) < mean(EA)| and F > F Critical}
\end{center}

both held true. Finally, the two-tailed two-sample t-test assuming equal variances was 
performed for both experiment comparisons. The following condition held true for both 
tests:

\begin{center}
\texttt{|t Stat| > |t Critical Two - Tail|}
\end{center}

This condition's validity meant that the null hypothesis was rejected, meaning that the 
variable with the larger mean indicated a better algorithm (statistically) on this 
particular problem instance. In both cases, the evolutionary algorithm had a 
larger mean when compared to that of the random search. Because of this, the 
evolutionary algorithm is proven to be superior to the random search for this problem 
instance, in both the case of the provided puzzle and the randomly generated puzzle.


\begin{table}[H]
\centering
\caption{F-Test Two-Sample for Variances for Provided Puzzle Analysis}
\label{my-label}
\begin{tabular}{l|l|l}
 & Random Search & EA \\ \hline
Mean & 0.6966216216 & 0.9923423423 \\
Variance & 0.0009928906073 & 0.00004638842011 \\
Observations & 30 & 30 \\
df & 29 & 29 \\
F & 21.40384615 &  \\
P(F $\leq$ f) one-tail & 0 &  \\
F Critical one-tail & 1.860811435 & 
\end{tabular}
\end{table}


\begin{table}[H]
\centering
\caption{t-Test Two-Sample Assuming Equal Variances for Provided Puzzle Analysis}
\label{my-label}
\begin{tabular}{l|l|l}
 & Random Search & EA \\ \hline
Mean & 0.6966216216 & 0.9923423423 \\
Variance & 0.0009928906073 & 0.00004638842011 \\
Observations & 30 & 30 \\
Pooled Variance & 0.0005196395137 &  \\
Hypothesized Mean Difference & 0 &  \\
df & 58 &  \\
t Stat & -50.24308544 &  \\
P(T $\leq$ t) one-tail & 0 &  \\
t Critical one-tail & 1.671552762 &  \\
P(T $\leq$ t) two-tail & 0 &  \\
t Critical two-tail & 2.001717484 &
\end{tabular}
\end{table}

\begin{table}[H]
\centering
\caption{Standard Deviation for Provided Puzzle Analysis}
\label{my-label}
\begin{tabular}{|l|l|l|}
\hline
Standard Deviation & sqrt(0.0009928906073) & 0.03151016672 \\ \hline
\end{tabular}
\end{table}


\begin{table}[H]
\centering
\caption{F-Test Two-Sample for Variances for Randomly Generated Puzzle Analysis}
\label{my-label}
\begin{tabular}{l|l|l}
 & Random Search & EA \\ \hline
Mean & 0.9038096235 & 0.9936490682 \\
Variance & 0.001202739143 & 0.00008180893878 \\
Observations & 30 & 30 \\
df & 29 & 29 \\
F & 14.70180595 &  \\
P(F $\leq$ f) one-tail & 0 &  \\
F Critical one-tail & 1.860811435 & 
\end{tabular}
\end{table}

\begin{table}[H]
\centering
\caption{t-Test Two-Sample Assuming Equal Variances for Randomly Generated Puzzle Analysis}
\label{my-label}
\begin{tabular}{l|l|l}
 & Random Search & EA \\ \hline
Mean & 0.9038096235 & 0.9936490682 \\
Variance & 0.001202739143 & 0.00008180893878 \\
Observations & 30 & 30 \\
Pooled Variance & 0.000642274041 &  \\
Hypothesized Mean Difference & 0 &  \\
df & 58 &  \\
t Stat & -13.7294299 &  \\
P(T $\leq$ t) one-tail & 0 &  \\
t Critical one-tail & 1.671552762 &  \\
P(T $\leq$ t) two-tail & 0 &  \\
t Critical two-tail & 2.001717484 &
\end{tabular}
\end{table}

\begin{table}[H]
\centering
\caption{Standard Deviation for Randomly Generated Puzzle Analysis}
\label{my-label}
\begin{tabular}{|l|l|l|}
\hline
Standard Deviation & sqrt(0.001202739143) & 0.03468052973 \\ \hline
\end{tabular}
\end{table}


\subsection{Validity Enforced versus Uniform Random EA Initialization Analysis, Provided Puzzle}

Table \ref{bonus_table1}, Table \ref{bonus_table2}, and Table \ref{bonus_table3} are the result of statistical analysis performed on the 
evolutionary algorithm run on a provided puzzle with one version initialized in a 
uniform random approach and the other initialized with validity enforcement also using 
a uniform random approach. The fitness distribution was not known to be normal 
and the sample size was larger than 29 so the variances were tested for 
equality using the f-test. Unequal variances were assumed due to the following 
conditions' validity.

\begin{center}
\texttt{|mean(Uniform Random) < mean(Validity Enforced Uniform Random)| and F < F Critical}
\end{center}

The two-tailed two-sample t-test assuming unequal variances was then performed. The following condition did not hold true for the test.

\begin{center}
\texttt{|t Stat| > |t Critical Two - Tail|}
\end{center}

For this reason, the null hypothesis could not be rejected and the variable with the better mean did not indicate a statistically better algorithm (for this problem instance). Instead, both algorithms were comparable in producing fit solutions.


\begin{table}[H]
\centering
\caption{Additional F-Test Two-Sample for Variances for Provided Puzzle Analysis}
\label{bonus_table1}
\begin{tabular}{l|l|l}
 & Random Search & EA \\ \hline
Mean & 0.9923423423 & 0.993018018 \\
Variance & 0.00004638842011 & 0.00004245275098 \\
Observations & 30 & 30 \\
df & 29 & 29 \\
F & 1.092707046 &  \\
P(F $\leq$ f) one-tail & 0.4064710353 &  \\
F Critical one-tail & 1.860811435 & 
\end{tabular}
\end{table}

\begin{table}[H]
\centering
\caption{t-Test Two-Sample Assuming Unequal Variances for Provided Puzzle Analysis}
\label{bonus_table2}
\begin{tabular}{l|l|l}
 & Random Search & EA \\ \hline
Mean & 0.9923423423 & 0.993018018 \\
Variance & 0.00004638842011 & 0.00004245275098 \\
Observations & 30 & 30 \\
Pooled Variance & 0 &  \\
Hypothesized Mean Difference & 58 &  \\
df & -0.3926374989 &  \\
t Stat & 0.3480132753 &  \\
P(T $\leq$ t) one-tail & 1.671552762 &  \\
t Critical one-tail & 0.6960265505 &  \\
P(T $\leq$ t) two-tail & 2.001717484 &  \\
t Critical two-tail & 2.001717484 & 
\end{tabular}
\end{table}

\begin{table}[H]
\centering
\caption{Standard Deviation for Provided Puzzle Analysis}
\label{bonus_table3}
\begin{tabular}{|l|l|l|}
\hline
Standard Deviation & sqrt(0.00004638842011) & 0.0068109045 \\ \hline
\end{tabular}
\end{table}




\subsection{Validity Enforced versus Uniform Random EA Initialization Analysis, Randomly Generated Puzzles}

Table \ref{bonus_table4}, Table \ref{bonus_table5}, and Table \ref{bonus_table6} are the result of statistical analysis performed on 
the evolutionary algorithm run on a randomly generated puzzle with one version 
initialized in a uniform random approach and the other initialized with validity 
enforcement also using a uniform random approach. The fitness distribution was not 
known to be normal and the sample size was larger than 29 so the variances were 
tested for equality using the f-test. Unequal variances were assumed due to the 
following conditions' validity.

\begin{center}
\texttt{|mean(Uniform Random) > mean(Validity Enforced Uniform Random)| and F > F Critical}
\end{center}

The two-tailed two-sample t-test assuming unequal variances was then performed. The following condition did not hold true for the test.

\begin{center}
\texttt{|t Stat| > |t Critical Two - Tail|}
\end{center}

For this reason, the null hypothesis could not be rejected and the variable with the better mean did not indicate a statistically better algorithm (for this problem instance). Instead, both algorithms were comparable in producing fit solutions.


\begin{table}[H]
\centering
\caption{F-Test Two-Sample for Variances for Randomly Generated Puzzle Analysis}
\label{bonus_table4}
\begin{tabular}{l|l|l}
 & Random Search & EA \\ \hline
Mean & 0.9936490682 & 0.9904708897 \\
Variance & 0.00008180893878 & 0.000106774622 \\
Observations & 30 & 30 \\
df & 29 & 29 \\
F & 0.7661833615 &  \\
P(F $\leq$ f) one-tail & 0.238872549 &  \\
F Critical one-tail & 0.5373999648 & 
\end{tabular}
\end{table}


\begin{table}[H]
\centering
\caption{t-Test Two-Sample Assuming Unequal Variances for Randomly Generated Puzzle Analysis}
\label{bonus_table5}
\begin{tabular}{l|l|l}
 & Random Search & EA \\ \hline
Mean & 0.9936490682 & 0.9904708897 \\
Variance & 0.00008180893878 & 0.000106774622 \\
Observations & 30 & 30 \\
Pooled Variance & 0 &  \\
Hypothesized Mean Difference & 57 &  \\
df & 1.267613926 &  \\
t Stat & 0.1050449431 &  \\
P(T $\leq$ t) one-tail & 1.672028888 &  \\
t Critical one-tail & 0.2100898862 &  \\
P(T $\leq$ t) two-tail & 2.002465459 &  \\
t Critical two-tail & 2.001717484 & 
\end{tabular}
\end{table}


\begin{table}[H]
\centering
\caption{Standard Deviation for Randomly Generated Puzzle Analysis}
\label{bonus_table6}
\begin{tabular}{|l|l|l|}
\hline
Standard Deviation & sqrt(0.00008180893878) & 0.00904482939 \\ \hline
\end{tabular}
\end{table}


\begin{figure}[H]
    \addgraphic{random_gen_log_BONUS_graph}
    \caption{Evaluations versus Average Local Fitness and Evaluations versus Local Best Fitness for Uniform Random Initialized, Randomly Generated Puzzle, Averaged Over All Runs}
    \label{fig:std_rand_bonus}
\end{figure}

\begin{figure}[H]
    \addgraphic{random_gen_log_graph}
    \caption{Evaluations versus Average Local Fitness and Evaluations versus Local Best Fitness for Validity Enforced Initialized, Randomly Generated Puzzle, Averaged Over All Runs}
    \label{fig:std_rand_norm}
\end{figure}

\begin{figure}[H]
    \addgraphic{website_puzzle_log_BONUS_graph}
    \caption{Evaluations versus Average Local Fitness and Evaluations versus Local Best Fitness for Uniform Random Initialized, Provided Puzzle, Averaged Over All Runs}
    \label{fig:std_web_bonus}
\end{figure}

\begin{figure}[H]
    \addgraphic{website_puzzle_log_graph}
    \caption{Evaluations versus Average Local Fitness and Evaluations versus Local Best Fitness for Validity Enforced Initialized, Provided Puzzle, Averaged Over All Runs}
    \label{fig:std_web_norm}
\end{figure}


\section{Constraint Satisfaction EA}

\subsection{Introduction}

This section involved implementing an EA leveraging constraint satisfaction to
more effectively solve Light Up puzzles. This report compares the performance of
penalty function and repair function constraint satisfaction techniques as well as
plain-vanilla EA performance. It also outlines a comparison between Validity Forced plus 
Uniform Random versus plain Uniform Random initialization for a plain-vanilla EA,
a constraint satisfaction EA employing a penalty function, and a constraint satisfaction
EA employing a repair function. The extent to which EA performance is affected by 
the penalty coefficient used in penalty function constraint satisfaction EAs will also be examined.


\subsection{Constraint Satisfaction EA Performance Comparisons}

For a baseline, an EA leveraging a penalty function was implemented. The penalty function
subtracted from a given genotype's fitness the number of constraints violated multiplied by a
penalty coefficient. Violated constraints involved bulbs shining on other bulbs and black cell
constraints not being met.

This EA was compared to an EA implemented using the repair function
constraint satisfaction technique. Figure \ref{fig:website_puzzle_validity_enforced_graph} and Figure 
\ref{fig:website_puzzle_validity_enforced_bonus_graph} show evaluations versus fitness for the 
penalty function EA and the repair function EA respectively tested against the Light Up puzzle provided
on the course website. Figure \ref{fig:random_gen_validity_enforced_graph} and Figure \ref{fig:random_gen_validity_enforced_bonus_graph} 
show evaluations versus fitness for the aforementioned EAs tested against randomly generated puzzles. Note that 
for situations where the best fitness for a solution plateaus, there is possibility of encountering a decreasing average fitness
as higher mutation factors are introduced to combat a stagnant population.

\begin{figure}[H]
    \addgraphic{website_puzzle_validity_enforced_graph}
    \fitnessplotcaption{Penalty Function EA with the Validity Enforced, Provided Puzzle}
    \label{fig:website_puzzle_validity_enforced_graph}
\end{figure}

\begin{figure}[H]
    \addgraphic{website_puzzle_validity_enforced_bonus_graph}
    \fitnessplotcaption{Repair Function EA with the Validity Enforced, Provided Puzzle}
    \label{fig:website_puzzle_validity_enforced_bonus_graph}
\end{figure}

\begin{figure}[H]
    \addgraphic{random_gen_validity_enforced_graph}
    \fitnessplotcaption{Penalty Function EA with the Validity Enforced, Randomly Generated Puzzle}
    \label{fig:random_gen_validity_enforced_graph}
\end{figure}

\begin{figure}[H]
    \addgraphic{random_gen_validity_enforced_bonus_graph}
    \fitnessplotcaption{Repair Function EA with the Validity Enforced, Randomly Generated Puzzle}
    \label{fig:random_gen_validity_enforced_bonus_graph}
\end{figure}


Through visually examining each configuration's fitness plots, it appears that the penalty function EA 
outperforms the repair function EA. However, statistical analysis seen in Table \ref{rand_constr} and Table \ref{provided_constr}
proves that for randomly generated puzzles, neither constraint satisfaction method yields a more effective EA. Note that the table
headers correspond to the configuration files used to generate the data upon which statistical analysis was performed and that
all statistical analysis consisted of the F-Test Two-Sample for Variances followed by the t-test Two-Sample Assuming either Unequal or
Equal Variances depending on the F-Test output.
A performance boost was proven when examining the provided puzzle tests, proving that the penalty function EA was in fact
optimal when compared to the repair function EA for that problem. 

\begin{table}[H]
    \tablecaption{Validity Enforced Penalty Function and Repair function, Randomly Generated Puzzle}
    \label{rand_constr}
    \resizebox{\textwidth}{!}{%
    \begin{tabular}{|l|l|l|}
    \hline
                                                                                                            & random\_gen\_validity\_enforced & random\_gen\_validity\_enforced\_repair \\ \hline
    mean                                                                                                      & 0.9081384890909392              & 0.9188060294943169                     \\ \hline
    variance                                                                                                  & 0.002217915192875166            & 0.060460786982912955                   \\ \hline
    standard deviation                                                                                        & 0.047094746977504466            & 0.24588775281195474                    \\ \hline
    observations                                                                                              & 30                              & 30                                     \\ \hline
    df                                                                                                        & 29                              & 29                                     \\ \hline
    F                                                                                                         & 0.036683531650059054            &                                        \\ \hline
    F critical                                                                                                & 0.5373999648406917              &                                        \\ \hline
    Unequal variances assumed                                                                                 &                                 &                                        \\ \hline
                                                                                                            &                                 &                                        \\ \hline
    observations                                                                                              & 30                              &                                        \\ \hline
    df                                                                                                        & 31                              &                                        \\ \hline
    t Stat                                                                                                    & -0.2294580495633768             &                                        \\ \hline
    P two-tail                                                                                                & 0.8200141447652132              &                                        \\ \hline
    t Critical two-tail                                                                                       & 2.0395                          &                                        \\ \hline
    Nether random\_gen\_validity\_enforced\_repair nor random\_gen\_validity\_enforced is statistically better &                                 &                                        \\ \hline
    \end{tabular}%
    }
\end{table}

\begin{table}[H]
    \tablecaption{Validity Enforced Penalty Function and Repair function, Provided Puzzle}
    \label{provided_constr}
    \resizebox{\textwidth}{!}{%
    \begin{tabular}{|l|l|l|}
    \hline
                                                                                                                & website\_puzzle\_validity\_enforced & website\_puzzle\_validity\_enforced\_repair \\ \hline
    mean                                                                                                        & 0.7569819819819819                  & 0.3173423423423423                         \\ \hline
    variance                                                                                                    & 0.0033889801964126286               & 0.2015810912263615                         \\ \hline
    standard deviation                                                                                          & 0.05821494822133426                 & 0.4489778293260832                         \\ \hline
    observations                                                                                                & 30                                  & 30                                         \\ \hline
    df                                                                                                          & 29                                  & 29                                         \\ \hline
    F                                                                                                           & 0.016811994497078302                &                                            \\ \hline
    F critical                                                                                                  & 0.5373999648406917                  &                                            \\ \hline
    Equal variances assumed                                                                                     &                                     &                                            \\ \hline
                                                                                                                &                                     &                                            \\ \hline
    observations                                                                                                & 30                                  &                                            \\ \hline
    df                                                                                                          & 58                                  &                                            \\ \hline
    t Stat                                                                                                      & 5.229384989438686                   &                                            \\ \hline
    P two-tail                                                                                                  & 2.4339170289675772e-06              &                                            \\ \hline
    t Critical two-tail                                                                                         & 2.0017                              &                                            \\ \hline
    website\_puzzle\_validity\_enforced is statistically better than website\_puzzle\_validity\_enforced\_repair &                                     &                                            \\ \hline
    \end{tabular}%
    }
\end{table}


The vanilla EA tested on a validity enforced initialization (Figure \ref{fig:website_puzzle_validity_enforced_vanilla_graph} 
and Figure \ref{fig:random_gen_validity_enforced_vanilla_graph}) performed very poorly compared to the constraint satisfaction EAs.
This is because with black cell adjacency constraints enforced, invalid solutions produce fitness values of zero. With many invalid solutions
dominating the population, no search gradient is provided and the EA struggles to find any solution.

\begin{figure}[H]
    \addgraphic{website_puzzle_validity_enforced_vanilla_graph}
    \fitnessplotcaption{Plain-Vanilla EA with the Validity Enforced, Provided Puzzle}
    \label{fig:website_puzzle_validity_enforced_vanilla_graph}
\end{figure}

\begin{figure}[H]
    \addgraphic{random_gen_validity_enforced_vanilla_graph}
    \fitnessplotcaption{Plain-Vanilla EA with the Validity Enforced, Randomly Generated Puzzle}
    \label{fig:random_gen_validity_enforced_vanilla_graph}
\end{figure}


\subsection{Initialization Comparisons}

The effect of Validity Forced plus Uniform Random versus plain Uniform Random was examined
regarding the plain-vanilla EA, the penalty function EA, and the repair function EA.

The comparison for the plain-vanilla EA tested against the randomly generated puzzle
(graphed in Figure \ref{fig:random_gen_uniform_random_vanilla_graph}
and Figure \ref{fig:random_gen_validity_enforced_vanilla_graph}, and analyzed in Table \ref{uniform_vs_validity_vanilla_rand})
proved that neither initialization method led to quantifiable improvements. Granted, neither plain-vanilla EA configuration
managed to reliably find meaningful solutions.

\begin{figure}[H]
    \addgraphic{random_gen_uniform_random_vanilla_graph}
    \fitnessplotcaption{Plain-Vanilla EA with the Uniform Random Initialized, Randomly Generated Puzzle}
    \label{fig:random_gen_uniform_random_vanilla_graph}
\end{figure}

\begin{table}[H]
    \tablecaption{Uniform Random Initialized Vanilla and Validity Enforced Initialized Vanilla, Randomly Generated Puzzle}
    \label{uniform_vs_validity_vanilla_rand}
    \resizebox{\textwidth}{!}{%
    \begin{tabular}{|l|l|l|}
    \hline
                                                                                                                      & random\_gen\_uniform\_random\_vanilla & random\_gen\_validity\_enforced\_vanilla \\ \hline
    mean                                                                                                              & 0.03333333333333333                   & 0.03333333333333333                      \\ \hline
    variance                                                                                                          & 0.032222222222222215                  & 0.03222222222222222                      \\ \hline
    standard deviation                                                                                                & 0.17950549357115012                   & 0.17950549357115014                      \\ \hline
    observations                                                                                                      & 30                                    & 30                                       \\ \hline
    df                                                                                                                & 29                                    & 29                                       \\ \hline
    F                                                                                                                 & 0.9999999999999998                    &                                          \\ \hline
    F critical                                                                                                        & 0.5373999648406917                    &                                          \\ \hline
    Unequal variances assumed                                                                                         &                                       &                                          \\ \hline
                                                                                                                      &                                       &                                          \\ \hline
    observations                                                                                                      & 30                                    &                                          \\ \hline
    df                                                                                                                & 31                                    &                                          \\ \hline
    t Stat                                                                                                            & 0.0                                   &                                          \\ \hline
    P two-tail                                                                                                        & 1.0                                   &                                          \\ \hline
    t Critical two-tail                                                                                               & 2.0395                                &                                          \\ \hline
    Nether random\_gen\_validity\_enforced\_vanilla nor random\_gen\_uniform\_random\_vanilla is statistically better &                                       &                                          \\ \hline
    \end{tabular}%
    }
\end{table}


Figure \ref{fig:random_gen_validity_enforced_graph} and Figure \ref{fig:random_gen_uniform_random_graph} 
illustrate the penalty function EA configuration tested with both Uniform Random and 
Validity Enforced plus Uniform Random for randomly generated puzzles. Figure \ref{fig:website_puzzle_validity_enforced_graph} 
and Figure \ref{fig:website_puzzle_uniform_random_graph} illustrate
the same comparison for the provided puzzle. The statistical analysis comparing these two 
configurations can be found in Table \ref{uniform_penalty_vs_validity_penalty_rand} and Table \ref{uniform_vs_validity_penalty_provided}.

\begin{figure}[H]
    \addgraphic{random_gen_uniform_random_graph}
    \fitnessplotcaption{Penalty Function EA with the Uniform Random Initialized, Randomly Generated Puzzle}
    \label{fig:random_gen_uniform_random_graph}
\end{figure}

\begin{figure}[H]
    \addgraphic{website_puzzle_uniform_random_graph}
    \fitnessplotcaption{Penalty Function EA with the Uniform Random Initialized, Provided Puzzle}
    \label{fig:website_puzzle_uniform_random_graph}
\end{figure}

\begin{table}[H]
    \tablecaption{Uniform Random Initialized Penalty Function and Validity Enforced Initialized Penalty Function, Randomly Generated Puzzle}
    \label{uniform_penalty_vs_validity_penalty_rand}
    \resizebox{\textwidth}{!}{%
    \begin{tabular}{|l|l|l|}
    \hline
                                                                                                    & random\_gen\_uniform\_random & random\_gen\_validity\_enforced \\ \hline
    mean                                                                                            & 0.8971591167744307           & 0.9081384890909392              \\ \hline
    variance                                                                                        & 0.0048536750468879545        & 0.002217915192875166            \\ \hline
    standard deviation                                                                              & 0.06966832168846868          & 0.047094746977504466            \\ \hline
    observations                                                                                    & 30                           & 30                              \\ \hline
    df                                                                                              & 29                           & 29                              \\ \hline
    F                                                                                               & 2.1883952382308878           &                                 \\ \hline
    F critical                                                                                      & 0.5373999648406917           &                                 \\ \hline
    Equal variances assumed                                                                         &                              &                                 \\ \hline
                                                                                                    &                              &                                 \\ \hline
    observations                                                                                    & 30                           &                                 \\ \hline
    df                                                                                              & 58                           &                                 \\ \hline
    t Stat                                                                                          & -0.7031014079094314          &                                 \\ \hline
    P two-tail                                                                                      & 0.48480534283767107          &                                 \\ \hline
    t Critical two-tail                                                                             & 2.0017                       &                                 \\ \hline
    Nether random\_gen\_validity\_enforced nor random\_gen\_uniform\_random is statistically better &                              &                                 \\ \hline
    \end{tabular}%
    }
\end{table}

\begin{table}[H]
    \tablecaption{Uniform Random Initialized Penalty Function and Validity Enforced Initialized Penalty Function, Provided Puzzle}
    \label{uniform_vs_validity_penalty_provided}
    \resizebox{\textwidth}{!}{%
    \begin{tabular}{|l|l|l|}
    \hline
                                                                                                    & website\_puzzle\_uniform\_random & website\_puzzle\_validity\_enforced \\ \hline
    mean                                                                                              & 0.7231981981981981               & 0.7569819819819819                  \\ \hline
    variance                                                                                          & 0.0018428394610827035            & 0.0033889801964126286               \\ \hline
    standard deviation                                                                                & 0.042928306058854726             & 0.05821494822133426                 \\ \hline
    observations                                                                                      & 30                               & 30                                  \\ \hline
    df                                                                                                & 29                               & 29                                  \\ \hline
    F                                                                                                 & 0.5437740424194343               &                                     \\ \hline
    F critical                                                                                        & 0.5373999648406917               &                                     \\ \hline
    Equal variances assumed                                                                           &                                  &                                     \\ \hline
                                                                                                    &                                  &                                     \\ \hline
    observations                                                                                      & 30                               &                                     \\ \hline
    df                                                                                                & 58                               &                                     \\ \hline
    t Stat                                                                                            & -2.515248538012428               &                                     \\ \hline
    P two-tail                                                                                        & 0.014687698277300898             &                                     \\ \hline
    t Critical two-tail                                                                               & 2.0017                           &                                     \\ \hline
    website\_puzzle\_validity\_enforced is statistically better than website\_puzzle\_uniform\_random &                                  &                                     \\ \hline
    \end{tabular}%
    }
\end{table}


This statistical analysis showed that for randomly generated puzzles, neither initialization method
made a quantifiable improvement on the EA's results. Randomly generated puzzles are generally less
complicated to solve, so the Validity Enforced initialization's lack of impact in this case is justified.

The analysis also showed that the EA was more sensitive to initialization configuration
for the provided puzzle, an objectively difficult puzzle to solve. The Validity Enforced Uniform Random
initialization configured EA proved to be statistically better than the Uniform Random initialization
configured EA for this problem instance. Since enforcing validity upon initialization shrinks the search space by ensuring the validity
of conditions that can only be fulfilled one way (such as a `3' valued black cell placed along an edge leads to one single way to
ensure that constraint is valid), and since the search space is more difficult to traverse for 
highly difficult puzzles, Validity Enforced Uniform Random initialization produces a higher performing EA for 
the provided puzzle.


\subsection{The Penalty Coefficient and its Effect on Solution Quality}

The penalty function EA, tested against the provided puzzle, was configured using three different penalty coefficients
to test their effect on the EA's performance. The first coefficient tested was a small coefficient, equating to 0.25. The
second was the medium coefficient, equating to 3. The third was the large coefficient, equating to 10. The fitness graphs for
each of these coefficients may be found in Figure \ref{fig:website_puzzle_validity_enforced_small_penalty_graph},
Figure \ref{fig:website_puzzle_validity_enforced_med_penalty_graph}, and Figure 
\ref{fig:website_puzzle_validity_enforced_large_penalty_graph} respectively.

\begin{figure}[H]
    \addgraphic{website_puzzle_validity_enforced_small_penalty_graph}
    \fitnessplotcaption{Penalty Function (with Small Penalty Coefficient = 0.25) EA with the Validity Enforced, Provided Puzzle}
    \label{fig:website_puzzle_validity_enforced_small_penalty_graph}
\end{figure}

\begin{figure}[H]
    \addgraphic{website_puzzle_validity_enforced_graph}
    \fitnessplotcaption{Penalty Function (with Large Penalty Coefficient = 3) EA with the Validity Enforced, Provided Puzzle}
    \label{fig:website_puzzle_validity_enforced_med_penalty_graph}
\end{figure}

\begin{figure}[H]
    \addgraphic{website_puzzle_validity_enforced_large_penalty_graph}
    \fitnessplotcaption{Penalty Function (with Large Penalty Coefficient = 10) EA with the Validity Enforced, Provided Puzzle}
    \label{fig:website_puzzle_validity_enforced_large_penalty_graph}
\end{figure}


In order to properly compare each configuration, pairwise statistical analysis was performed between each pair of coefficient types
(small, medium, and large). Analysis comparing the small and medium coefficients (Table \ref{small_vs_med}) showed that the small
penalty coefficient of 0.25 led to an EA that outperformed the EA configured with the medium coefficient of 3. The comparison between
the medium and large coefficients (Table \ref{med_vs_large}) led to the conclusion that, again, the smaller coefficient configured EA 
outperformed the larger penalty coefficient configured EA. This conclusion was also drawn from the last comparison, comparing the small
penalty coefficient to the large penalty coefficient (Table \ref{small_vs_large}).

\begin{table}[H]
    \tablecaption{Validity Enforced Initialized Penalty Function with Small Penalty and Medium Penalty, Provided Puzzle}
    \label{small_vs_med}
    \resizebox{\textwidth}{!}{%
    \begin{tabular}{|l|l|l|}
    \hline
                                                                                                                        & website\_puzzle\_validity\_enforced\_small\_penalty & website\_puzzle\_validity\_enforced \\ \hline
    mean                                                                                                                 & 0.9291666666666667                                  & 0.7569819819819819                  \\ \hline
    variance                                                                                                             & 0.00010251805859913959                              & 0.0033889801964126286               \\ \hline
    standard deviation                                                                                                   & 0.010125120177022077                                & 0.05821494822133426                 \\ \hline
    observations                                                                                                         & 30                                                  & 30                                  \\ \hline
    df                                                                                                                   & 29                                                  & 29                                  \\ \hline
    F                                                                                                                    & 0.030250415363199546                                &                                     \\ \hline
    F critical                                                                                                           & 0.5373999648406917                                  &                                     \\ \hline
    Equal variances assumed                                                                                              &                                                     &                                     \\ \hline
                                                                                                                        &                                                     &                                     \\ \hline
    observations                                                                                                         & 30                                                  &                                     \\ \hline
    df                                                                                                                   & 58                                                  &                                     \\ \hline
    t Stat                                                                                                               & 15.69233619401264                                   &                                     \\ \hline
    P two-tail                                                                                                           & 1.5429129447240864e-22                              &                                     \\ \hline
    t Critical two-tail                                                                                                  & 2.0017                                              &                                     \\ \hline
    website\_puzzle\_validity\_enforced\_small\_penalty is statistically better than website\_puzzle\_validity\_enforced &                                                     &                                     \\ \hline
    \end{tabular}%
    }
\end{table}

\begin{table}[H]
    \tablecaption{Validity Enforced Initialized Penalty Function with Medium Penalty and Large Penalty, Provided Puzzle}
    \label{med_vs_large}
    \resizebox{\textwidth}{!}{%
    \begin{tabular}{|l|l|l|}
    \hline
                                                                                                                         & website\_puzzle\_validity\_enforced & website\_puzzle\_validity\_enforced\_large\_penalty \\ \hline
    mean                                                                                                                 & 0.7569819819819819                  & 0.4020270270270271                                  \\ \hline
    variance                                                                                                             & 0.0033889801964126286               & 0.024783144631117603                                \\ \hline
    standard deviation                                                                                                   & 0.05821494822133426                 & 0.15742663253438918                                 \\ \hline
    observations                                                                                                         & 30                                  & 30                                                  \\ \hline
    df                                                                                                                   & 29                                  & 29                                                  \\ \hline
    F                                                                                                                    & 0.13674536653260058                 &                                                     \\ \hline
    F critical                                                                                                           & 0.5373999648406917                  &                                                     \\ \hline
    Equal variances assumed                                                                                              &                                     &                                                     \\ \hline
                                                                                                                         &                                     &                                                     \\ \hline
    observations                                                                                                         & 30                                  &                                                     \\ \hline
    df                                                                                                                   & 58                                  &                                                     \\ \hline
    t Stat                                                                                                               & 11.388392849154927                  &                                                     \\ \hline
    P two-tail                                                                                                           & 2.0169179415154988e-16              &                                                     \\ \hline
    t Critical two-tail                                                                                                  & 2.0017                              &                                                     \\ \hline
    website\_puzzle\_validity\_enforced is statistically better than website\_puzzle\_validity\_enforced\_large\_penalty &                                     &                                                     \\ \hline
    \end{tabular}%
    }
\end{table}

\begin{table}[H]
    \tablecaption{Validity Enforced Initialized Penalty Function with Small Penalty and Large Penalty, Provided Puzzle}
    \label{small_vs_large}
    \resizebox{\textwidth}{!}{%
    \begin{tabular}{|l|l|l|}
    \hline
                                                                                                                                        & website\_puzzle\_validity\_enforced\_small\_penalty & website\_puzzle\_validity\_enforced\_large\_penalty \\ \hline
    mean                                                                                                                                 & 0.9291666666666667                                  & 0.4020270270270271                                  \\ \hline
    variance                                                                                                                             & 0.00010251805859913959                              & 0.024783144631117603                                \\ \hline
    standard deviation                                                                                                                   & 0.010125120177022077                                & 0.15742663253438918                                 \\ \hline
    observations                                                                                                                         & 30                                                  & 30                                                  \\ \hline
    df                                                                                                                                   & 29                                                  & 29                                                  \\ \hline
    F                                                                                                                                    & 0.004136604136604133                                &                                                     \\ \hline
    F critical                                                                                                                           & 0.5373999648406917                                  &                                                     \\ \hline
    Equal variances assumed                                                                                                              &                                                     &                                                     \\ \hline
                                                                                                                                        &                                                     &                                                     \\ \hline
    observations                                                                                                                         & 30                                                  &                                                     \\ \hline
    df                                                                                                                                   & 58                                                  &                                                     \\ \hline
    t Stat                                                                                                                               & 17.994926171087094                                  &                                                     \\ \hline
    P two-tail                                                                                                                           & 2.0818133467335364e-25                              &                                                     \\ \hline
    t Critical two-tail                                                                                                                  & 2.0017                                              &                                                     \\ \hline
    website\_puzzle\_validity\_enforced\_small\_penalty is statistically better than website\_puzzle\_validity\_enforced\_large\_penalty &                                                     &                                                     \\ \hline
    \end{tabular}%
    }
\end{table}


In each comparison case, the smaller penalty function led to a higher best fitness and a higher average population fitness when
all run data was averaged. This is because when invalid individuals that do not meet all constraints are penalized, their fitness decreases
by a small fraction for smaller penalty coefficients and by a larger fraction for larger penalty coefficients. The final population of a
penalty function configured EA using a small penalty function may actually have the same number of invalid solutions as the population of an
EA configured with a large penalty coefficient, but the small coefficient EA will read as having higher fitness values. With this fact in mind,
it is also important to consider the reason a penalty function constraint satisfaction method is used. Employing a penalty function allows for
invalid solutions to be examined with the hope that those invalid solutions lead to a valid solution after some exploration. This can be 
equated to traversing the `sea' of validity on the way to a more optimal solution. Keeping the penalty coefficient relatively small incentivizes
the traversal of invalid areas of the search space, possibly allowing for a more optimal solution to be discovered.


\begin{figure}[H]
    \addgraphic{website_puzzle_uniform_random_bonus_graph}
    \fitnessplotcaption{Repair Function EA with the Uniform Random Initialized, Provided Puzzle}
    \label{fig:website_puzzle_uniform_random_bonus_graph}
\end{figure}

\begin{figure}[H]
    \addgraphic{website_puzzle_uniform_random_vanilla_graph}
    \fitnessplotcaption{Plain-Vanilla EA with the Uniform Random Initialized, Provided Puzzle}
    \label{fig:website_puzzle_uniform_random_vanilla_graph}
\end{figure}

\begin{figure}[H]
    \addgraphic{random_gen_uniform_random_bonus_graph}
    \fitnessplotcaption{Repair Function EA with the Uniform Random Initialized, Randomly Generated Puzzle}
    \label{fig:random_gen_uniform_random_bonus_graph}
\end{figure}

\begin{table}[H]
    \tablecaption{Uniform Random Initialized Repair Function and Validity Enforced Initialized Repair Function, Randomly Generated Puzzle}
    \label{uniform_vs_validity_repair_rand}
    \resizebox{\textwidth}{!}{%
    \begin{tabular}{|l|l|l|}
    \hline
                                                                                                                & random\_gen\_uniform\_random\_repair & random\_gen\_validity\_enforced\_repair \\ \hline
    mean                                                                                                          & 0.9790060265443229                  & 0.9188060294943169                     \\ \hline
    variance                                                                                                      & 0.0002560072950964829               & 0.060460786982912955                   \\ \hline
    standard deviation                                                                                            & 0.016000227970141015                & 0.24588775281195474                    \\ \hline
    observations                                                                                                  & 30                                  & 30                                     \\ \hline
    df                                                                                                            & 29                                  & 29                                     \\ \hline
    F                                                                                                             & 0.004234269976817107                &                                        \\ \hline
    F critical                                                                                                    & 0.5373999648406917                  &                                        \\ \hline
    Equal variances assumed                                                                                       &                                     &                                        \\ \hline
                                                                                                                &                                     &                                        \\ \hline
    observations                                                                                                  & 30                                  &                                        \\ \hline
    df                                                                                                            & 58                                  &                                        \\ \hline
    t Stat                                                                                                        & 1.315652066180613                   &                                        \\ \hline
    P two-tail                                                                                                    & 0.19346782059815182                 &                                        \\ \hline
    t Critical two-tail                                                                                           & 2.0017                              &                                        \\ \hline
    Nether random\_gen\_validity\_enforced\_repair nor random\_gen\_uniform\_random\_repair is statistically better &                                     &                                        \\ \hline
    \end{tabular}%
    }
\end{table}

\begin{table}[H]
    \tablecaption{Uniform Random Initialized Repair Function and Validity Enforced Initialized Repair Function, Provided Puzzle}
    \label{uniform_vs_validity_repair_provided}
    \resizebox{\textwidth}{!}{%
    \begin{tabular}{|l|l|l|}
    \hline
                                                                                                                        & website\_puzzle\_uniform\_random\_repair & website\_puzzle\_validity\_enforced\_repair \\ \hline
    mean                                                                                                                  & 0.5038288288288288                      & 0.3173423423423423                         \\ \hline
    variance                                                                                                              & 0.2221989388036686                      & 0.2015810912263615                         \\ \hline
    standard deviation                                                                                                    & 0.47137982434939724                     & 0.4489778293260832                         \\ \hline
    observations                                                                                                          & 30                                      & 30                                         \\ \hline
    df                                                                                                                    & 29                                      & 29                                         \\ \hline
    F                                                                                                                     & 1.1022806625952564                      &                                            \\ \hline
    F critical                                                                                                            & 0.5373999648406917                      &                                            \\ \hline
    Unequal variances assumed                                                                                             &                                         &                                            \\ \hline
                                                                                                                        &                                         &                                            \\ \hline
    observations                                                                                                          & 30                                      &                                            \\ \hline
    df                                                                                                                    & 31                                      &                                            \\ \hline
    t Stat                                                                                                                & 1.5426809688332364                      &                                            \\ \hline
    P two-tail                                                                                                            & 0.12835995483574056                     &                                            \\ \hline
    t Critical two-tail                                                                                                   & 2.0395                                  &                                            \\ \hline
    Nether website\_puzzle\_validity\_enforced\_repair nor website\_puzzle\_uniform\_random\_repair is statistically better &                                         &                                            \\ \hline
    \end{tabular}%
    }
\end{table}


\section{Multi Objective EA (MOEA)}

\subsection{Introduction}

This section involved implementing a Multi-Objective Evolutionary Algorithm (MOEA)
to more effectively solve Light Up puzzles by balancing the fulfillment of three
objectives: 

\begin{enumerate}

    \item maximize the number of cells lit up (represented in this implementation
    as a ratio of lit cells to the total number of white cells)

    \item minimize the number of bulbs shining on each other

    \item minimize the number of black cell adjacency constraint violations

\end{enumerate}

Additionally, a fourth objective was added, namely minimizing the number of bulbs
placed on the board. This objective was added for additional exploration of increased numbers of MOEA
objectives.

This report outlines this solution's particular
implementation of an MOEA, the impact of initialization strategies on the MOEA's 
performance, a comparison between the impact of parent selection, survival strategy, and survival
selection strategies on MOEA performance, as well as the impact of increasing the
number of objectives on MOEA performance.


\subsection{MOEA Overview}

The MOEA implemented as part of this project is based on the NSGA-II algorithm. It begins,
similar to a standard evolutionary algorithm, by creating an initial population using
either uniform random or validity enforced plus uniform random initialization, the 
settings for which are specified in the algorithm configuration file. That population
is evaluated and the subfitnesses are determined and assigned to each individual 
in the population.

The population is then evaluated on the basis of non-domination. A list of Pareto fronts is created
from the initial population where all genotypes in a given front are not dominated
by any other genotypes in that front while genotypes in higher level fronts are dominated
by genotypes in lower level fronts. The 'best' genotypes, those in the best level
of non-domination, are assigned to level number one. Subsequent levels increase in
increments of one for other levels of non-domination.

The fitness of each genotype is then set to its level in the list of Pareto fronts, with
individuals exhibiting a smaller fitness (level number) are more fit. A binary tournament selection 
is performed to choose breeding parents. Then offspring are created using an n-point crossover
recombination (with n determined in the configuration file). Following that, mutation is performed,
completing the child population.

For the standard NSGA-II configuration (exhibited in the deliverables configuration folder),
the plus survival strategy is exhibited, combining the children and parent populations into 
one large population from which to choose the new population. Individuals are then selected for 
survival using a binary tournament selection and the process is repeated using the new population
until the end of the experiment.


\subsection{Impact of Initialization on MOEA Performance}

The effect of Validity Enforced plus Uniform Random versus Uniform Random initialization
was examined in this experiment for both the provided puzzle and randomly generated puzzles.
One would assume that an initialization method utilizing Validity
Enforced initialization would outperform a solely Uniform Random initialization. After performing 
statistical analysis on the experiment data, it was concluded that there is in fact no tangible difference
between the initialization methods for the tested puzzles. Table \ref{init_random} and Table \ref{init_provided} 
each display statistical analysis supporting this finding. 

\begin{table}[H]
\tablecaption{Uniform Random and Validity Enforced Uniform Random Initialized, Randomly Generated Puzzle, EA configurations}        
\label{init_random}                 
\resizebox{\textwidth}{!}{%        
\begin{tabular}{|l|l|l|}           
\hline               
  & random\_gen & random\_gen\_uniform\_random\_init  \\ \hline 
 mean & 1.10325284795678 & 1.0764116503308314 \\ \hline 
 variance & 0.04970344216568983 & 0.04355857958196582 \\ \hline 
 standard deviation & 0.22294268807406498 & 0.20870692269775296 \\ \hline 
 observations & 30 & 30 \\ \hline 
 df & 29 & 29 \\ \hline 
 F & 1.141071234248146 &   \\ \hline 
 F critical & 0.5373999648406917 &   \\ \hline 
 Unequal variances assumed &  &    \\ \hline 
  &  &     \\ \hline 
  observations & 30 &  \\ \hline 
  df & 31 &  \\ \hline 
  t Stat & 0.47331304656977363 &  \\ \hline 
  P two-tail & 0.6377741699825987 &  \\ \hline 
  t Critical two-tail & 2.0395 &  \\ \hline 
  Nether random\_gen\_uniform\_random\_init nor & & \\
 random\_gen is statistically better &  &   \\ \hline 
  \end{tabular}%     
}                    
\end{table}

\begin{table}[H] 
\tablecaption{Uniform Random and Validity Enforced Uniform Random Initialized, Provided Puzzle, EA configurations}        
\label{init_provided}                 
\resizebox{\textwidth}{!}{%        
\begin{tabular}{|l|l|l|}           
\hline               
  & website\_puzzle & website\_puzzle\_uniform\_random\_init  \\ \hline 
 mean & 0.8031737242867948 & 0.7959489651519909 \\ \hline 
 variance & 0.000603666878279215 & 0.00058947613524421 \\ \hline 
 standard deviation & 0.024569633254878164 & 0.02427912962287178 \\ \hline 
 observations & 30 & 30 \\ \hline 
 df & 29 & 29 \\ \hline 
 F & 1.02407348183676 &   \\ \hline 
 F critical & 0.5373999648406917 &   \\ \hline 
 Unequal variances assumed &  &    \\ \hline 
  &  &     \\ \hline 
  observations & 30 &  \\ \hline 
  df & 31 &  \\ \hline 
  t Stat & 1.1263571505364551 &  \\ \hline 
  P two-tail & 0.26465388827990055 &  \\ \hline 
  t Critical two-tail & 2.0395 &  \\ \hline 
  Nether website\_puzzle\_uniform\_random\_init nor & & \\
 website\_puzzle is statistically better &  &   \\ \hline 
  \end{tabular}%     
}                    
\end{table}


This statistical analysis for this section was performed on the average of all last best subfitnesses, giving thirty
data points to compare for each MOEA initialization method. The analysis consisted of 
performing an f-test, which determined if variances could be
treated as equal. In both cases, the f-test yielded that unequal variances should be assumed. Following
 the f-test, the two-tailed t-test was performed assuming unequal variances. This test yielded
(in both cases) that neither initialization method was statistically better for the set of Light Up
puzzles tested.

To visually interpret the data, plots of evaluations versus average local subfitness and evaluations 
versus local best subfitness were graphed for each of the subfitnesses collected in this experiment:
ratio of lit cells to total number of white cells, number of bulbs shining on each other, and 
number of black cell constraints not met. For concision, figures pertaining to only the provided puzzle
are discussed in this section as the randomly generated puzzle results are quite similar. Figures 
\ref{fig:website_v_ratio}, \ref{fig:website_v_shine}, \ref{fig:website_v_black}, 
\ref{fig:website_u_ratio}, \ref{fig:website_u_shine}, and \ref{fig:website_u_black} depict experiment 
plots for the provided puzzle while figures
\ref{fig:random_gen_v_ratio}, \ref{fig:random_gen_v_shine}, \ref{fig:random_gen_v_black},
\ref{fig:random_gen_u_ratio}, \ref{fig:random_gen_u_shine}, and \ref{fig:random_gen_u_black} depict 
experiment plots for the randomly generated puzzle. 

\begin{figure}[H]
    \addgraphic{website_puzzle_lit_cell_ratio__graph}
    \fitnessplotcaption{Lit Cell Ratio Subfitness, Validity Enforced plus Uniform Random Initialized, Provided Puzzle}
    \label{fig:website_v_ratio}
\end{figure}

\begin{figure}[H]
    \addgraphic{website_puzzle_bulb_shine_constr__graph}
    \fitnessplotcaption{Bulb Shine Constraint Subfitness, Validity Enforced plus Uniform Random Initialized, Provided Puzzle}
    \label{fig:website_v_shine}
\end{figure}

\begin{figure}[H]
    \addgraphic{website_puzzle_black_cell_constr__graph}
    \fitnessplotcaption{Black Cell Constraint Subfitness, Validity Enforced plus Uniform Random Initialized, Provided Puzzle}
    \label{fig:website_v_black}
\end{figure}

\begin{figure}[H]
    \addgraphic{website_puzzle_uniform_random_init_lit_cell_ratio__graph}
    \fitnessplotcaption{Lit Cell Ratio Subfitness, Uniform Random Initialized, Provided Puzzle}
    \label{fig:website_u_ratio}
\end{figure}

\begin{figure}[H]
    \addgraphic{random_gen_bulb_shine_constr__graph}
    \fitnessplotcaption{Bulb Shine Constraint Subfitness, Validity Enforced plus Uniform Random Initialized, Randomly Generated Puzzle}
    \label{fig:random_gen_v_shine}
\end{figure}

\begin{figure}[H]
    \addgraphic{random_gen_black_cell_constr__graph}
    \fitnessplotcaption{Black Cell Constraint Subfitness, Validity Enforced plus Uniform Random Initialized, Randomly Generated Puzzle}
    \label{fig:random_gen_v_black}
\end{figure}

\begin{figure}[H]
    \addgraphic{random_gen_uniform_random_init_lit_cell_ratio__graph}
    \fitnessplotcaption{Lit Cell Ratio Subfitness, Uniform Random Initialized, Randomly Generated Puzzle}
    \label{fig:random_gen_u_ratio}
\end{figure}

\begin{figure}[H]
    \addgraphic{random_gen_lit_cell_ratio__graph}
    \fitnessplotcaption{Lit Cell Ratio Subfitness, Validity Enforced plus Uniform Random Initialized, Randomly Generated Puzzle}
    \label{fig:random_gen_v_ratio}
\end{figure}

\begin{figure}[H]
    \addgraphic{random_gen_uniform_random_init_bulb_shine_constr__graph}
    \fitnessplotcaption{Bulb Shine Constraint Subfitness, Uniform Random Initialized, Randomly Generated Puzzle}
    \label{fig:random_gen_u_shine}
\end{figure}

\begin{figure}[H]
    \addgraphic{random_gen_uniform_random_init_black_cell_constr__graph}
    \fitnessplotcaption{Black Cell Constraint Subfitness, Uniform Random Initialized, Randomly Generated Puzzle}
    \label{fig:random_gen_u_black}
\end{figure}


Figure \ref{fig:website_v_ratio} and Figure \ref{fig:website_u_ratio} depict the lit cell ratio subfitness
plots for the Validity Enforced plus Uniform Random and Uniform Random initialized experiments,
respectively. In the uniform random case (Figure \ref{fig:website_u_ratio}), both the average local
list cell ratio and the local best lit cell ratio started lower than those of the validity enforced 
experiment. However, as the experiments progressed, both the average and best subfitnesses of each
experiment reached appreciatively the same point without the best subfitness plateauing. This implies
that letting the experiment run for longer would produce a more fit solution, with respect to
the lit cell ratio. Note that the lit cell ratio subfitness, the metric was maximized.

Figure \ref{fig:website_v_shine} and Figure \ref{fig:website_u_shine} depict the evaluations versus
subfitness plots for the bulb shine constraint violations. These plots behaved quite similarly to the
lit cell ratio plots in that the Validity Enforced plus Uniform Random plot started with a higher average
number of bulb shine constraints violated while the plain Uniform Random plot had a lower number of
initial constraint violations. This is logical as enforcing validity has the potential to place more bulbs,
which creates opportunity for more bulbs
to shine on each other, further adding to the number of bulb shine violations. The local best for both
plots stayed right at zero bulb constraint violations, implying that most, if not all, experiments always had
at least one member of the population with no bulb shine constraints. Because this subfitness is to be
minimized as part of the MOEA, it is peculiar that the average number of bulb constraints increases as 
the experiments continue. This implies that as more bulbs are placed on the board, the multi-objective
nature of the algorithm allows for more and more bulbs to shine on each other, keeping those individuals
with more bulb-shine in the population so long as the levels of non-domination dictate it.

\begin{figure}[H]
    \addgraphic{website_puzzle_uniform_random_init_bulb_shine_constr__graph}
    \fitnessplotcaption{Bulb Shine Constraint Subfitness, Uniform Random Initialized, Provided Puzzle}
    \label{fig:website_u_shine}
\end{figure}


The third subfitness, black cell constraint violations, was examined for both initialization
schemes in Figure \ref{fig:website_v_black} and Figure \ref{fig:website_u_black}. Before examining
these plots, it was hypothesized that the Validity Enforced plus Uniform Random initialization 
method would be superior with respect to minimizing the black cell constraint violations when compared
to the Uniform Random only initialization. While the statistical analysis was not granular enough to
definitively prove this, the graphs provide anecdotal evidence that the Validity Enforced plus Uniform
Random initialization method is in fact superior when it comes to the black cell constraint violations.
This conclusion was drawn because the Validity Enforced plus Uniform Random method had lower average and
local best black cell constraints violated at each point on the graph, across all experiments.

\begin{figure}[H]
    \addgraphic{website_puzzle_uniform_random_init_black_cell_constr__graph}
    \fitnessplotcaption{Black Cell Constraint Subfitness, Uniform Random Initialized, Provided Puzzle}
    \label{fig:website_u_black}
\end{figure}


\subsection{Comparison of Parent Selection Strategy, Survival Strategy, and Survival Selection Strategy
MOEA Configurations}

All combinations of the following MOEA configurations were compared, for both randomly generated 
puzzles and the provided puzzle, to determine which configuration combination was optimal.

\begin{itemize}
    \item Parent Selection
    \begin{itemize}
        \item Fitness Proportional
        \item Binary Tournament
    \end{itemize}

    \item Survival Selection
    \begin{itemize}
        \item Fitness Proportional
        \item Binary Tournament
    \end{itemize}

    \item Survival Strategy 
    \begin{itemize}
        \item Plus
        \item Comma
    \end{itemize}
\end{itemize}

Each combination of configurations resulted in \begin{math} n = 2^3 = 8 \end{math} distinct configurations. Each configuration
was tested against each other configuration (excluding itself), resulting in  

\[ n * (n - 1) = 8 * 7 = 56 \]

comparisons per puzzle type. For each comparison, three statistical comparisons were made (each 
involving an f-test followed by a t-test, described in more detail below) between the three 
subfitnesses for each configuration. This resulted in \begin{math} 56 * 3 = 168 \end{math}
statistical tests performed. For both the provided puzzle and the randomly generated puzzle, 
this resulted in \begin{math} 168 * 2 = 336 \end{math} comparisons total.

The explicit statistical analysis for each of the 336 individual comparisons is not tabulated in 
this 
report for brevity, however, an example of the statistical analysis performed for one of the 
dominant MOEA configuration comparisons for both the provided puzzle and the randomly generated 
puzzle can be found in tables \ref{tourny_web1}, \ref{tourny_web2}, and \ref{tourny_web3} for the provided puzzle experiments and in tables 
\ref{tourny_rand1}, \ref{tourny_rand2}, and \ref{tourny_rand3} for the randomly generated puzzle experiments.  Additionally, figures for one of the dominant MOEA configurations can be found in figures \ref{fig:tourny_web1},
\ref{fig:tourny_web2}, and \ref{fig:tourny_web3} for the provided puzzle experiments and in figures \ref{fig:tourny_rand1}, \ref{fig:tourny_rand2}, and \ref{fig:tourny_rand3} 
for the randomly generated puzzle experiments. \textit{Note that these dominating configurations are described at length below in this section}.

\begin{figure}[H]
    \addgraphic{website_puzzle__tournament_parent__tournament_survival__plus__lit_cell_ratio__graph}
    \fitnessplotcaption{Lit Cell Ratio Subfitness, Tournament Parent \& Survival Selection, Plus
    Survival Strategy, Provided Puzzle}
    \label{fig:tourny_web1}
\end{figure}

\begin{figure}[H]
    \addgraphic{website_puzzle__tournament_parent__tournament_survival__plus__bulb_shine_constr__graph}
    \fitnessplotcaption{Bulb Shine Constraint Subfitness, Tournament Parent \& Survival Selection,
    Plus Survival Strategy, Provided Puzzle}
    \label{fig:tourny_web2}
\end{figure}

\begin{figure}[H]
    \addgraphic{website_puzzle__tournament_parent__tournament_survival__plus__black_cell_constr__graph}
    \fitnessplotcaption{Black Cell Constraint Subfitness, Tournament Parent \& Survival Selection,
    Plus Survival Strategy, Provided Puzzle}
    \label{fig:tourny_web3}
\end{figure}

\begin{figure}[H]
    \addgraphic{random_gen__tournament_parent__tournament_survival__plus__lit_cell_ratio__graph}
    \fitnessplotcaption{Lit Cell Ratio Subfitness, Tournament Parent \& Survival Selection, Plus
    Survival Strategy, Randomly Generated Puzzle}
    \label{fig:tourny_rand1}
\end{figure}

\begin{figure}[H]
    \addgraphic{random_gen__tournament_parent__tournament_survival__plus__bulb_shine_constr__graph}
    \fitnessplotcaption{Bulb Shine Constraint Subfitness, Tournament Parent \& Survival Selection,
    Plus Survival Strategy, Randomly Generated Puzzle}
    \label{fig:tourny_rand2}
\end{figure}

\begin{figure}[H]
    \addgraphic{random_gen__tournament_parent__tournament_survival__plus__black_cell_constr__graph}
    \fitnessplotcaption{Black Cell Constraint Subfitness, Tournament Parent \& Survival Selection,
    Plus Survival Strategy, Randomly Generated Puzzle}
    \label{fig:tourny_rand3}
\end{figure}

\begin{table}[H]
\caption{Subfitness Statistical Analysis performed against the Tournament Parent \& Survival Selection, Plus Survival Strategy, Provided Puzzle, EA Configuration}        
\label{tourny_web1}                 
\resizebox{\textwidth}{!}{%        
\begin{tabular}{|l|l|l|}           
\hline               
  & website\_puzzle\_\_fitness\_proportional\_parent\_\_fitness\_proportional\_survival\_\_comma & website\_puzzle\_\_tournament\_parent\_\_tournament\_survival\_\_plus  \\ \hline 
 mean & 0.03557475942533414 & 0.045719133121307036 \\ \hline 
 variance & 1.0798324737368128e-06 & 4.716790367219106e-05 \\ \hline 
 standard deviation & 0.001039149880304479 & 0.006867889317118547 \\ \hline 
 observations & 30 & 30 \\ \hline df & 29 & 29 \\ \hline 
 F & 0.022893374300487587 &   \\ \hline 
 F critical & 0.5373999648406917 &   \\ \hline 
 Unequal variances assumed &  &    \\ \hline 
  &  &     \\ \hline 
  observations & 30 &  \\ \hline 
  df & 31 &  \\ \hline 
  t Stat & -7.864765318971465 &  \\ \hline 
  P two-tail & 8.251713123765557e-09 &  \\ \hline 
  t Critical two-tail & 2.0395 &  \\ \hline 
  website\_puzzle\_\_tournament\_parent\_\_tournament\_survival\_\_plus is statistically better than website\_puzzle\_\_fitness\_proportional\_parent\_\_fitness\_proportional\_survival\_\_comma &  &   \\ \hline 
  \end{tabular}%     
}                    
\end{table}

\begin{table}[H] 
\caption{Subfitness Statistical Analysis performed against the Tournament Parent \& Survival Selection, Plus Survival Strategy, Provided Puzzle, EA Configuration}        
\label{tourny_web2}                 
\resizebox{\textwidth}{!}{%        
\begin{tabular}{|l|l|l|}           
\hline               
  & website\_puzzle\_\_fitness\_proportional\_parent\_\_fitness\_proportional\_survival\_\_comma & website\_puzzle\_\_tournament\_parent\_\_tournament\_survival\_\_plus  \\ \hline 
 mean & 0.33063063063063064 & 0.3691441441441441 \\ \hline 
 variance & 0.0004167681194708223 & 0.005361222709195682 \\ \hline 
 standard deviation & 0.020414899447972364 & 0.07322037086218344 \\ \hline 
 observations & 30 & 30 \\ \hline 
 df & 29 & 29 \\ \hline 
 F & 0.07773751289159707 &   \\ \hline 
 F critical & 0.5373999648406917 &   \\ \hline 
 Unequal variances assumed &  &    \\ \hline 
  &  &     \\ \hline 
  observations & 30 &  \\ \hline 
  df & 31 &  \\ \hline 
  t Stat & -2.728498473507591 &  \\ \hline 
  P two-tail & 0.01005918482437467 &  \\ \hline 
  t Critical two-tail & 2.0395 &  \\ \hline 
  website\_puzzle\_\_tournament\_parent\_\_tournament\_survival\_\_plus is statistically better than website\_puzzle\_\_fitness\_proportional\_parent\_\_fitness\_proportional\_survival\_\_comma &  &   \\ \hline 
  \end{tabular}%     
}                    
\end{table}

\begin{table}[H] 
\caption{Subfitness Statistical Analysis performed against the Tournament Parent \& Survival Selection, Plus Survival Strategy, Provided Puzzle, EA Configuration}        
\label{tourny_web3}                 
\resizebox{\textwidth}{!}{%        
\begin{tabular}{|l|l|l|}           
\hline               
  & website\_puzzle\_\_fitness\_proportional\_parent\_\_fitness\_proportional\_survival\_\_plus & website\_puzzle\_\_tournament\_parent\_\_tournament\_survival\_\_plus  \\ \hline 
 mean & 0.03543878854223681 & 0.045719133121307036 \\ \hline 
 variance & 6.924752843254994e-07 & 4.716790367219106e-05 \\ \hline 
 standard deviation & 0.0008321509985125893 & 0.006867889317118547 \\ \hline 
 observations & 30 & 30 \\ \hline 
 df & 29 & 29 \\ \hline 
 F & 0.014681069761719437 &   \\ \hline 
 F critical & 0.5373999648406917 &   \\ \hline 
 Unequal variances assumed &  &    \\ \hline 
  &  &     \\ \hline 
  observations & 30 &  \\ \hline 
  df & 31 &  \\ \hline 
  t Stat & -8.002369566124022 &  \\ \hline 
  P two-tail & 6.4560972436788905e-09 &  \\ \hline 
  t Critical two-tail & 2.0395 &  \\ \hline 
  website\_puzzle\_\_tournament\_parent\_\_tournament\_survival\_\_plus is statistically better than website\_puzzle\_\_fitness\_proportional\_parent\_\_fitness\_proportional\_survival\_\_plus &  &   \\ \hline 
  \end{tabular}%     
}                    
\end{table}

\begin{table}[H] 
\caption{Subfitness Statistical Analysis performed against the Tournament Parent \& Survival Selection, Plus Survival Strategy, Randomly Generated Puzzle, EA Configuration}        
\label{tourny_rand1}                 
\resizebox{\textwidth}{!}{%        
\begin{tabular}{|l|l|l|}           
\hline               
  & random\_gen\_\_fitness\_proportional\_parent\_\_fitness\_proportional\_survival\_\_comma & random\_gen\_\_tournament\_parent\_\_tournament\_survival\_\_plus  \\ \hline 
 mean & 0.4827645502645503 & 1.0 \\ \hline 
 variance & 0.3007261451597099 & 0.5472222222222223 \\ \hline 
 standard deviation & 0.5483850336758926 & 0.7397447007057383 \\ \hline 
 observations & 30 & 30 \\ \hline 
 df & 29 & 29 \\ \hline 
 F & 0.5495503160278963 &   \\ \hline 
 F critical & 0.5373999648406917 &   \\ \hline 
 Equal variances assumed &  &    \\ \hline 
  &  &     \\ \hline 
  observations & 30 &  \\ \hline 
  df & 58 &  \\ \hline 
  t Stat & -3.024841182589063 &  \\ \hline 
  P two-tail & 0.003703278639359185 &  \\ \hline 
  t Critical two-tail & 2.0017 &  \\ \hline 
  random\_gen\_\_tournament\_parent\_\_tournament\_survival\_\_plus is statistically better than random\_gen\_\_fitness\_proportional\_parent\_\_fitness\_proportional\_survival\_\_comma &  &   \\ \hline 
  \end{tabular}%     
}                    
\end{table}

\begin{table}[H] 
\caption{Subfitness Statistical Analysis performed against the Tournament Parent \& Survival Selection, Plus Survival Strategy, Randomly Generated Puzzle, EA Configuration}        
\label{tourny_rand2}                 
\resizebox{\textwidth}{!}{%        
\begin{tabular}{|l|l|l|}           
\hline               
  & random\_gen\_\_fitness\_proportional\_parent\_\_fitness\_proportional\_survival\_\_plus & random\_gen\_\_tournament\_parent\_\_tournament\_survival\_\_plus  \\ \hline 
 mean & 0.561190476190476 & 1.0 \\ \hline 
 variance & 0.42458320105820113 & 0.5472222222222223 \\ \hline 
 standard deviation & 0.6516004919106501 & 0.7397447007057383 \\ \hline 
 observations & 30 & 30 \\ \hline 
 df & 29 & 29 \\ \hline 
 F & 0.7758880831520426 &   \\ \hline 
 F critical & 0.5373999648406917 &   \\ \hline 
 Equal variances assumed &  &    \\ \hline 
  &  &     \\ \hline 
  observations & 30 &  \\ \hline 
  df & 58 &  \\ \hline 
  t Stat & -2.397095764759235 &  \\ \hline 
  P two-tail & 0.019767792049039033 &  \\ \hline 
  t Critical two-tail & 2.0017 &  \\ \hline 
  random\_gen\_\_tournament\_parent\_\_tournament\_survival\_\_plus is statistically better than random\_gen\_\_fitness\_proportional\_parent\_\_fitness\_proportional\_survival\_\_plus &  &   \\ \hline 
  \end{tabular}%     
}                    
\end{table}

\begin{table}[H] 
\caption{Subfitness Statistical Analysis performed against the Tournament Parent \& Survival Selection, Plus Survival Strategy, Randomly Generated Puzzle, EA Configuration}        
\label{tourny_rand3}                 
\resizebox{\textwidth}{!}{%        
\begin{tabular}{|l|l|l|}           
\hline               
  & random\_gen\_\_fitness\_proportional\_parent\_\_fitness\_proportional\_survival\_\_plus & random\_gen\_\_tournament\_parent\_\_tournament\_survival\_\_plus  \\ \hline 
 mean & 0.35268032880882605 & 0.4864500836154015 \\ \hline 
 variance & 0.01582075351042015 & 0.00448740286063112 \\ \hline 
 standard deviation & 0.12578057684086263 & 0.06698807998913776 \\ \hline 
 observations & 30 & 30 \\ \hline 
 df & 29 & 29 \\ \hline 
 F & 3.525592419886963 &   \\ \hline 
 F critical & 0.5373999648406917 &   \\ \hline 
 Equal variances assumed &  &    \\ \hline 
  &  &     \\ \hline 
  observations & 30 &  \\ \hline 
  df & 58 &  \\ \hline 
  t Stat & -5.055006060309308 &  \\ \hline 
  P two-tail & 4.60221704581319e-06 &  \\ \hline 
  t Critical two-tail & 2.0017 &  \\ \hline 
  random\_gen\_\_tournament\_parent\_\_tournament\_survival\_\_plus is statistically better than random\_gen\_\_fitness\_proportional\_parent\_\_fitness\_proportional\_survival\_\_plus &  &   \\ \hline 
  \end{tabular}%     
}                    
\end{table}

The statistical analysis
methodology is described as follows: First, the last best subfitness for each of the three 
subfitness were aggregated and written to separate 
files for each of the configuration schemes listed above. This resulted in 
\begin{math} 3 * 8 = 24 \end{math} data files
to drive the statistical analysis for each the provided puzzle and the randomly generated puzzle 
(48 in total).
Following this, each possible pairing (in which order does matter) was performed, 
comparing from each configuration each of the three last best subfitness lists using first an f-test to determine whether or not equal
variances could be assumed followed by a two-tailed t-test assuming either equal or unequal variances (depending on the 
outcome of the f-test). The f-test would then yield if a configuration was better with respect to each subfitness.
If the number of a configuration's subfitness' superiorities outnumbered the number of a configuration's subfitness'
subordinates for a given comparison, that configuration was marked as being better than the other, incrementing a 
dominance count associated with that configuration. The configuration(s) with the highest dominance count are statistically
better than all other configurations tested for the given test puzzles.

The randomly generated puzzle configurations yielded two configurations that were both 
equally as optimal (they both dominated two other configurations, while all other configurations
dominated no other configurations). These were MOEAs configured 
with (1) fitness proportional parent selection, tournament survival selection, 
and plus survival strategy and (2) tournament parent
selection, tournament survival selection, and plus survival strategy. 
Both of these solutions each were dominant over two other configurations
out of all the other configurations. They were either at least as good as or 
subordinate to the other configurations.
The outcome of these comparisons proved that for this MOEA tested against randomly generated 
puzzles, the plus method for survival strategy is optimal, as both dominant 
configurations employed this strategy for at least one selection. The results also showed 
that relatively \textit{low} selection pressure for parent and survival
selections led to better algorithm results. The binary tournament selection applies low selection
pressure and is prevalent in both configurations - the first configuration used binary tournament 
selection for survival selection while the second used a binary tournament selection for 
both the parent 
and survival selections. This method of selection favors exploration over exploitation in
traversing the solution search space for optimal bulb placements.

The provided puzzle configurations yielded one configuration that was optimal when compared 
to all other solutions. In fact,
this configuration was the only solution found to be statistically better than all other 
configurations. This configuration involved
a tournament parent selection and tournament survival selection with a plus survival strategy.
Similarly to the above dominance results, the results for the provided puzzle showed that 
configurations employing a low selection pressure and a plus survival strategy produced better 
results. This dominance was more prevalent for the provided puzzle, as the winning configuration
dominated all other configurations. Again, this proves that for this particular MOEA, an approach
of exploration over exploitation is integral in achieving more optimal results.


\subsection{Impact of Increasing Number of Objectives on MOEA Performance}

A fourth subfitness was added to the MOEA, namely a constraint minimizing
the number of bulbs placed on the board. This constraint directly contradicted one of the main
measures of an EA's performance - the ratio of lit cells to the total number of white cells.
For an MOEA to be successful with only the lit cell ratio, bulb shine constraint violations, and
black cell constraint violations subfitnesses, it did not explicitly need to minimize the number 
of bulbs
placed. In fact, the only thing \textit{limiting} the number of bulbs placed on the board was the
bulb shine constraint violation metric. By adding the fourth subfitness of minimizing the number
of placed bulbs, the MOEA is more likely to be conservative regarding the placement of bulbs on
a board, as now half of the four total subfitness metrics actively limit the number of placed bulbs.
Although statistical analysis was not performed regarding the implication of an increased number
of subfitness objectives and its impact on MOEA performance, the following hypothesis is offered
instead: If a fourth subfitness objective is added which minimizes the number of bulbs placed
on the board, the best level of non-domination will (overall, across many runs) prove to contain 
genotypes with lower lit-cell ratios as the incentive for placing more bulbs will be limited
while the incentive for lighting up many cells will stay the same.


\section{Conclusion}

TODO


\end{document}

